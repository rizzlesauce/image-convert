\documentclass[11pt]{article}  % required first line, though can vary;
                               % this says we will use 11-point font,
                               % in the "article" format

\usepackage{graphicx}
\usepackage{verbatim}

% material beginning with the percent sign is commentary, for human
% information purposes, not processed by the LaTeX system

% these \setlength etc. lines concern page layout, amount of paragraph
% indentation etc.; beginners should ignore them (but include them)
\setlength{\oddsidemargin}{0.0in}
\setlength{\evensidemargin}{0.0in}
\setlength{\topmargin}{-0.25in}
\setlength{\headheight}{0in}
\setlength{\headsep}{0in}
\setlength{\textwidth}{6.5in}
\setlength{\textheight}{9.25in}
\setlength{\parindent}{0in}
\setlength{\parskip}{2mm}

\begin{document}  % required; the document starts here

\title{Project 1 Writeup and Paper-Pencil Problems}

\author{Ross Adamson \\
\\
        Sec. 1}

\maketitle

\section{Paper-Pencil Problems}

1. A standard music CD holds 74 minutes of stereo (2-channel) audio at 44,100 samples/second and 16 bits/sample/channel. How many megabytes of raw data can such a CD hold?

% the $ delimiter marks the start and end of a mathematical expression
\[= \frac{mb}{1024kb} \times \frac{kb}{1024byte} \times \frac{byte}{8bit}
\times \frac{16bit}{channel} \times \frac{2channel}{sample}
\times \frac{44,100sample}{sec} \times \frac{60sec}{min}
\times 74min\]
\[\approx 746.93mb\]

2. Book problems

2.5 How many line pairs per mm will this camera be able to resolve?

\includegraphics[width=.8\textwidth]{diagram}

\[\frac{d}{500mm} = \frac{7mm}{35mm}\]
\[d = 500mm \times \frac{7mm}{35mm}\]
\[= 100mm\]

\[\frac{linepear}{mm} = \frac{linepear}{2line}
\times \frac{1024line}{100mm}\]
\[= 5.12 \frac{linepears}{mm}\]

2.10 How many bits would it take to store a 2-hour HDTV movie?

\[l = 1125\]
\[\frac{16}{9} = \frac{col}{l}\]
\[col = 1125 \times \frac{16}{9} = 2000\]
\[px = col \times l = 1125 \times 2000 = 2,250,000px\]

Every image has 2,250,000 pixels. Half those pixels are required
every 1/60th of a second (1,125,000 pixels). So, total pixels required
for an hour is:

\[\frac{1,125,000px}{frame} \times \frac{60frame}{sec} \times
\frac{60sec}{min} \times \frac{60min}{hr}\]
\[= 243 \times 10^9px\]

For two hours that's $486 \times 10^9$ pixels. Every pixel has 24 bits, 
so the number of bits required for a 2 hour
HDTV movie is $11,664 \times 10^9$ bits.

2.19 Show that an operator that computes the median of a subimage
area, S, is nonlinear.

Let med be an operator that computes the median of a subimage area. Assume, 
in order to form a contradiction, that med is linear. Consider the 
following two images

$
f_1 =
\left[ {\begin{array}{cc}
 4 & 2  \\
 1 & 3  \\
 \end{array} } \right]
$ and $
f_2 =
\left[ {\begin{array}{cc}
 2 & 5  \\
 4 & 3  \\
 \end{array} } \right]
$

and suppose that we let $a_1 = 1$ and $a_2 = -1$. To test for linearity,
we start with the left side of Eq.(2.6-2):

\[
med
\left\{ {
(1)
\left[ {\begin{array}{cc}
 4 & 2  \\
 1 & 3  \\
 \end{array} } \right]
+
(-1)
\left[ {\begin{array}{cc}
 2 & 5  \\
 4 & 3  \\
 \end{array} } \right] }
\right\}
=
med
\left\{ {
\left[ {\begin{array}{cc}
 2 & -3  \\
 -3 & 0  \\
 \end{array} } \right] }
\right\}
\]
\[= -0.5\]

Working next with the right side, we obtain

\[
(1)
med
\left\{ {
    \left[ {\begin{array}{cc}
     4 & 2  \\
     1 & 3  \\
     \end{array} } \right]
} \right\}
+
(-1)
med
\left\{ {
    \left[ {\begin{array}{cc}
     2 & 5  \\
     4 & 3  \\
     \end{array} } \right]
} \right\}
= 2.5 + (-1)3.5
\]
\[= -1\]

The left and right sides of the equation are not equal in this case, so
we have proved that in general the med operator is nonlinear.

\section{Project Writeup}

I used Python and the PIL library for this project.

\begin{figure}
\centering
\includegraphics[scale=.8]{parrots}
\caption{Color image converted to grayscale}
\label{grayscale}
\end{figure}

\begin{figure}
\centering
\includegraphics[angle=-90, width=.8\textwidth]{histogram}
\caption{Histogram of grayscale distribution}
\label{histogram}
\end{figure}

% a blank line means a new paragraph

\end{document}  % required; the document ends here

